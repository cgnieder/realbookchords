\begin{filecontents}{\jobname.bib}
@online{web:pietsch:fonts,
  author  = {Jochen Pietsch} ,
  title   = {Jazz Fonts} ,
  url     = {http://notation.jochenpietsch.de/index_e.html} ,
  urldate = {2012-05-10}
}
@book{book:realbook,
  author    = {Various} ,
  title     = {The Real Book Volume I} ,
  subtitle  = {C Edition} ,
  publisher = {Hal Leonard Publishing Corporation} ,
  isbn      = {978-0634060380} ,
  date      = {2000-01-01} ,
  edition   = {6th edition}
}
@book{book:newrealbook,
  author    = {Various} ,
  title     = {The New Real Book Volume I} ,
  subtitle  = {C Edition} ,
  publisher = {Ama Verlag} ,
  isbn      = {978-0961470142} ,
  date      = {2009-02-12} ,
  edition   = {1st edition}
}
\end{filecontents}
\begin{filecontents}{index_en.ist}
 preamble "\\begin{theindex}\n Section titles are indicated \\textbf{bold}, packages \\textsf{sans serif}, commands \\code{\\textbackslash\\textcolor{code}{brown}} and options \\textcolor{key}{\\code{green}}.\\newline\n\n"
 heading_prefix "{\\bfseries "
 heading_suffix "\\hfil}\\nopagebreak\n"
 headings_flag  1
 delim_0 "\\dotfill "
 delim_1 "\\dotfill "
 delim_2 "\\dotfill "
 delim_r "\\nohyperpage{\\textendash}"
 suffix_2p "\\nohyperpage{\\,f.}"
 suffix_3p "\\nohyperpage{\\,ff.}"
\end{filecontents}
\documentclass[toc=index,toc=bib]{cnpkgdoc}
\usepackage{musixtex}
\let\breve\relax
\docsetup{
  pkg      = realbookchords ,
  code-box = {skipabove=1ex,skipbelow=1ex}
}
\addcmds{
  addspace,
  afterruleskip,
  AtBeginDocument,
  b,
  bar,
  bb,
  bk,
  breve,
  ca,
  doublebar,
  dt,
  en,
  endpiece,
  generalmeter,
  ha,hlp,isluru,
  k,
  kb,
  kk,
  mbreve,
  meterfrac,
  newfontfamily,
  NewRealBookChords,
  nobarnumbers,
  notes,Notes,NOtes,NOTes,NOTEs,
  qlp,qu,
  rbc,
  RBCsetup,
  RealBookChords,
  RealBookTitle,
  setsign,
  startpiece,
  textordmasculine,
  tslur,
  Uptext
}
\usepackage{polyglossia}
  \setmainlanguage{english}

\newfontfamily\RealBookTitle{Real Book Title}
\newfontfamily\RealBookChords{RealBook Chords}

\usepackage{hologo}
\usepackage{csquotes}

% save musixtex's definition of \addspace
\let\maddspace\addspace
\let\addspace\relax
\usepackage[backend=biber,style=alphabetic]{biblatex}
% restore musixtex's definition of \addspace
\let\baddspace\addspace
\let\addspace\maddspace
\addbibresource{\jobname.bib}

\let\verb\lstinline
\makeindex
\begin{document}

\section{License and Requirements}
\realbookchords is placed under the terms of the LaTeX Project Public License,
version 1.3 or later (\url{http://www.latex-project.org/lppl.txt}).
It has the status \enquote{maintained}.

In order to function properly a rather up to date version of the \paket{l3kernel}
is needed. \realbookchords also needs \paket*{xparse} (part of the \paket{l3packages}
bundle) and \paket{fontspec} for proper font support.

\realbookchords does not offer the fonts it uses. You need to install them on
your system by yourself and then use this package with \hologo{XeLaTeX} or
\hologo{LuaLaTeX} to be able to use its features.

\section{About}
On his website Jochen Pietsch offers three fonts \cite{web:pietsch:fonts} which
mimic the styles of the famous The Real Book\footnote{The original illegal edition
is no longer available, though. I am uncertain if the 6th edition still uses the
same handwritten font.} \cite{book:realbook} and its successor The New Real Book
\cite{book:newrealbook}:

\begin{itemize}
 \item New Real Book Chords
 \item Real Book Title
 \item RealBook Chords
\end{itemize}

\realbookchords does \emph{not} offer these fonts. Instead it offers easy to use macros
to use the first font with \hologo{XeLaTeX} or \hologo{LuaLaTeX} if it is installed
on your system. Especially a macro for the intuitive usage of the sometimes rather
hidden symbols for the creation of chord symbols is offered.

\section{Setup}
\realbookchords defines a few options which can be globally set using
\begin{beschreibung}
 \Befehl{RBCsetup}{<options>}
\end{beschreibung}
The options follow a key/value system like in many other \LaTeX\ packages.

\section{The Fonts}
Now first let's take a look at the fonts. You can get them at \url{http://notation.jochenpietsch.de/}.
They are placed under the Creative Commons Attribution-Noncommercial-Share Alike
3.0 Unported License. Jochen Pietsch who designed the fonts says there that he
doesn't develop the fonts any further so we have to live with any shortcomings:

\begin{zitat}[Jochen Pietsch]
 Please note, these fonts are still in beta-stage, so you might note some missing
 characters, or you may need to tweak the layout. Because my interest in this
 project has decreased, I have no plans to do any further improvements.
\end{zitat}

The license of the fonts forbids to use them for commercial purposes. This is not
strictly true for this package but as it is rather useless without them, well\ldots

This documentation defines new font families for two of the three fonts, the third
one is defined by \realbookchords.
\begin{beispiel}
 % from the preamble of this document
 \newfontfamily\RealBookTitle{Real Book Title}
 \newfontfamily\RealBookChords{RealBook Chords}
\end{beispiel}

The package provides this font family switch:
\begin{beschreibung}
 \befehl{NewRealBookChords} activate the font family \enquote{New Real Book Chords}
\end{beschreibung}

\subsection{New Real Book Chords}
The most important font for this package is \enquote{New Real Book Chords}. The
font has only a limited number of characters since its purpose is to build chords.
This also means that it has some unexpected symbols. But have a look for yourself:
\begin{beispiel}
 \normalsize\NewRealBookChords ABCDEFGHIJKLMNOPQRSTUVWXYZ \\
 a cdefghijklmnopqrstuvwxyz 0123456789 \\
 .,:!/*="'\&()[]+-\textordmasculine\% \\
 b\`{a}\ss{} \#{}\~{a}? < > ; {\makeatletter @} \_ \\
 \'{a} \aa{} \"{a} \^{a} \`{A} \'{A} \^{A} \~{A} \"{A} \AA{} \AE{} \"{E} \`{E} \'{E}
\end{beispiel}

\subsection{Real Book Title}
The following example is simply to show you the font, \realbookchords does not
use it at all.
\begin{beispiel}
 \normalsize\RealBookTitle ABCDEFGHIJKLMNOPQRSTUVWXYZ \\
 abcdefghijklmnopqrstuvwxyz 01234 6 -/\_
\end{beispiel}

\subsection{RealBook Chords}
The following example is simply to show you the font, \realbookchords does not
use it at all.
\begin{beispiel}
 \normalsize\RealBookChords ABCDEFG ab s 01 34567 9 \#()+-\textasciicircum
\end{beispiel}

\section{Shortcomings}
Since the font is missing some symbols, for instance the uppercase delta for a
maj7 chord or the striked through o (similar to ø) for a half-diminished chord,
for one thing this package does workarounds and else we have to live with it.

There are plans for the future to extend this package for the use with the more
professional font available here: \url{http://www.jazzfont.com/}. But for that
I'll first have to be able to buy the font and experiment with it a bit.
\emph{Finis coronat opus}.

\section{Typesetting Chords}
It's time to get to the important stuff: the chords. To typeset them there is one
basic command:
\begin{beschreibung}
 \befehl{rbc}[<options>]{<chord specs>}
\end{beschreibung}
The \ma{<chord specs>} will be explained in detail in the rest of the documentation.

\subsection{Basics}
The basic usage is pretty self-explanatory:
\begin{beispiel}
 \rbc{Ab9} \rbc{Cmi} \rbc{E+} \rbc{Gma9} \rbc{F\#mi9} \rbc{Db13}
\end{beispiel}
Note that this is nearly but not exactly the same as using the font directly:
\begin{beispiel}
 \NewRealBookChords Ab9 Cmi E+ Gma9 F\#mi9 Db13
\end{beispiel}
The font provides the characters \rbc{>} and \rbc{<}. \cmd{jc} replaces the strings
\code{mi} and \code{ma} with them. You have also seen that a \code{b} gives \rbc{b}
for a flat root and \verb+\#+ gives \rbc{\#} for a sharp root.

\subsection{Extensions}
One very important aspect of \enquote{jazzy} chords is tensions. The basic tensions
are clear -- just insert the intervall number:
\begin{beispiel}
 \rbc{G7} \rbc{A9} \rbc {F11} \rbc{E13} \rbc{B7+}
\end{beispiel}
Often enough one needs alterated extensions. \realbookchords defines two macros
to access the characters \rbc{\b} and \rbc{\k} easily:
\begin{beschreibung}
 \befehl{b} minor/diminished extension
 \befehl{k} major/augmented extension
\end{beschreibung}
 Let's see them in action:
\begin{beispiel}
  \rbc{Fmi7(\b5)} \rbc{G7(\k9)} \rbc{Eb7(\b9)} \rbc{Db9(\k11)} \rbc{Cmi7(5-)}
\end{beispiel}

There are also six \enquote{extension descriptions}:
\begin{beispiel}
 \rbc{Cmaj9} \rbc{Bbadd9} \rbc{Gsus4} \rbc{Ealt} \rbc{F7omit3} \rbc{D\#dim} \\
\end{beispiel}
Since they're not all available as single characters, \realbookchords fakes the
missing ones. This results in inconsistent looks. That's why \realbookchords
provides an option so that all six are faked:
\begin{beschreibung}
 \option{use-fake-symbols}{\default{true}/false} switch between original characters and faked ones. Default = \code{false}
\end{beschreibung}
\begin{beispiel}
 \RBCsetup{use-fake-symbols}
 \rbc{Cmaj9} \rbc{Bbadd9} \rbc{Gsus4} \rbc{Ealt} \rbc{F7omit3} \rbc{D\#dim}
\end{beispiel}

\subsection{Double Extensions}
Sometimes more than one alterated extension needs to be indicated. The
\enquote{New Real Book Chords} font provides a number of characters for this
purpose. \realbookchords provides five macros to access them easily:
\begin{beschreibung}
 \befehl{bb}{<intervalls>} \verb=\rbc{\bb{13,9}}= \rbc{\bb{13,9}}
 \befehl{bk}{<intervalls>} \verb=\rbc{\bk{13,9}}= \rbc{\bk{13,9}}
 \befehl{kb}{<intervalls>} \verb=\rbc{\kb{13,9}}= \rbc{\kb{13,9}}
 \befehl{kk}{<intervalls>} \verb=\rbc{\kk{13,9}}= \rbc{\kk{13,9}}
 \befehl{dt}[<alterations>]{<intervalls>} \verb=\dt[bb]{13,9}= \dt[bb]{13,9}
\end{beschreibung}
There are four different combinations of alterations. For each there is a macros.
The fifth macro also provides a possibility to access to the for combinations but
also enables to only alterate only one of the two extensions. Unlike the first
four macros \cmd{dt} is \emph{not only inside} \cmd{rbc} defined butr can also be
used in normal text.

All possible \oa{<alterations>} are shown below:
\begin{beispiel}
 \dt{13,9} \dt[bb]{13,9} \dt[bk]{13,9} \dt[kb]{13,9} \dt[kk]{13,9} \dt[B]{13,9} \dt[K]{13,9} \dt[b]{13,9} \dt[k]{13,9}
\end{beispiel}

The intervall numbers for both extensions are given seperated with a comma. If the
combination is present as a character like for example \rbc[double-extensions-brackets=false]{\dt{13,9}}
then the character is used else it is faked: \rbc[double-extensions-brackets=false]{\dt{9,6}}

\begin{beschreibung}
 \option{fake-double-extensions}{\default{true}/false} when \code{true} \emph{all} double extension numbers are faked. Default = \code{false}
 \option{double-extensions-brackets}{\default{true}/false} enclose the double extensions in brackets or not. Default = \code{false}
\end{beschreibung}
There is a shortcut for the \key{double-extensions-brackets} option: \cmd[rbca]{rbc*}. The brackets can also be set explicitly using \code{[} and \code{]}.
\begin{beispiel}
 \rbc{Eb7\bk{9,5}} \rbc[double-extensions-brackets=false]{C\dt{9,6}} \rbc*{C\dt{9,6}} \rbc*{C[\bk{13,9}]}
\end{beispiel}

\subsection{Bass Notes}
If you want to indicate a different bass note for the chord you simply seperate the
main chord from the bass note with a slash. Please note: if the bass note is a
flat or sharp note you have to enclose it in braces:
\begin{beispiel}
 \rbc{Cmi7/{Bb}} \rbc{C7/G} \rbc{C/G} \rbc{E7/{G\#}}
\end{beispiel}

\subsection{Disabling the Parsing}
If you don't want all the parsing of the \cmd{rbc} command but simply access the
\enquote{New Real Book Chords} font you can of course use the font family switch
presented earlier:
\begin{beispiel}
 \NewRealBookChords Ama7(omit3) [major]
\end{beispiel}

You can also use the following option:
\begin{beschreibung}
 \option{parse}{\default{true}/false} switch off the parsing. Default = \code{true}
\end{beschreibung}
\begin{beispiel}
 \texttt{parse=true}: \rbc{Ama7(omit3) [major]} \\
 \texttt{parse=false}: \rbc[parse=false]{Ama7(omit3) [major]}
\end{beispiel}

\section{See it in Action}
Let's use the \cmd{rbc} command together with \paket{musixtex} for a real example.
Both \paket{fontspec} which is loaded be \realbookchords and \paket{musixtex}
define the command \cmd{breve}. The definition of \paket{fontspec} takes place
\verb=\AtBeginDocument=.

You can do something like the following to get them to work together:
\begin{beispiel}[code only]
 \documentclass{article}
 \usepackage{musixtex}
 \let\mbreve\breve
 \let\breve\relax
 \usepackage{realbookchords}
 \begin{document}
  do stuff, e.g. restore the `musixtex' definition of \breve
 \end{document}
\end{beispiel}

A similar problem arises when \paket{musixtex} is used together with \paket{biblatex}:
they both define \cmd{addspace}.

Let's take the first eight bars of Kenny Durham's \enquote{Blue Bossa} as an example:

\begin{beispiel}[code only]
\begin{music}
 \parindent0pt \generalmeter{\meterfrac44}\setsign{1}{-3}\nobarnumbers
 \startpiece
 \addspace{.5\afterruleskip}%
 \notes\Uptext{\small\rbc{Bossa}}\en
 \NOtes\qa g\en
 \doublebar % 1
 \NOtes\Uptext{\rbc{Cmi6}}\qlp n\en
 \Notes\ca l\en
 \Notes\qa k\en
 \Notes\isluru0j\ca j\en
 \bar % 2
 \NOtes\tslur0j\hlp j\en
 \NOtes\qa i\en
 \bar % 3
 \NOTes\Uptext{\rbc{Fmi7}}\ha h\en
 \Notes\qlp n\en
 \Notes\isluru0m\ca m\en
 \bar % 4
 \NOTEs\tslur0m\wh m\en
 \bar % 5
 \NOtes\Uptext{\rbc{Dmi7(\b5)}}\qlp m\en
 \Notes\ca{lk}\qa j\en
 \Notes\isluru0i\ca i\en
 \bar % 6
 \NOTEs\Uptext{\rbc{G7(\k5)}}\tslur0i\hlp i\en
 \Notes\qa h\en
 \bar % 7
 \NOTes\Uptext{\rbc{Cmi6}}\ha g\en
 \Notes\qlp m\isluru0l\ca l\en
 \bar % 8
 \NOTEs\tslur0l\wh l\en
 \endpiece
\end{music}
\end{beispiel}

\begin{music}
 \parindent0pt \generalmeter{\meterfrac44}\setsign{1}{-3}\nobarnumbers
 \startpiece
 \addspace{.5\afterruleskip}%
 \notes\Uptext{\small\rbc{Bossa}}\en
 \NOtes\qa g\en
 \doublebar % 1
 \NOtes\Uptext{\rbc{Cmi6}}\qlp n\en
 \Notes\ca l\en
 \Notes\qa k\en
 \Notes\isluru0j\ca j\en
 \bar % 2
 \NOtes\tslur0j\hlp j\en
 \NOtes\qa i\en
 \bar % 3
 \NOTes\Uptext{\rbc{Fmi7}}\ha h\en
 \Notes\qlp n\en
 \Notes\isluru0m\ca m\en
 \bar % 4
 \NOTEs\tslur0m\wh m\en
 \bar % 5
 \NOtes\Uptext{\rbc{Dmi7(\b5)}}\qlp m\en
 \Notes\ca{lk}\qa j\en
 \Notes\isluru0i\ca i\en
 \bar % 6
 \NOTEs\Uptext{\rbc{G7(\k5)}}\tslur0i\hlp i\en
 \Notes\qa h\en
 \bar % 7
 \NOTes\Uptext{\rbc{Cmi6}}\ha g\en
 \Notes\qlp m\isluru0l\ca l\en
 \bar % 8
 \NOTEs\tslur0l\wh l\en
 \endpiece
\end{music}

\appendix
\let\addspace\baddspace
\printbibliography
\printindex
\end{document}