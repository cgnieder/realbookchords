% arara: lualatex: { action: nonstopmode }
% --------------------------------------------------------------------------
% the REALBOOKCHORDS package
% 
%   Typesetting jazz chords the `Real Book' way
% 
% --------------------------------------------------------------------------
% Clemens Niederberger
% Web:    https://bitbucket.org/cgnieder/realbookchords/
% E-Mail: contact@mychemistry.eu
% --------------------------------------------------------------------------
% Copyright 2012--2017 Clemens Niederberger
% 
% This work may be distributed and/or modified under the
% conditions of the LaTeX Project Public License, either version 1.3
% of this license or (at your option) any later version.
% The latest version of this license is in
%   http://www.latex-project.org/lppl.txt
% and version 1.3 or later is part of all distributions of LaTeX
% version 2005/12/01 or later.
% 
% This work has the LPPL maintenance status `maintained'.
% 
% The Current Maintainer of this work is Clemens Niederberger.
% --------------------------------------------------------------------------
% If you have any ideas, questions, suggestions or bugs to report, please
% feel free to contact me.
% --------------------------------------------------------------------------
\documentclass[load-preamble+]{cnltx-doc}

\let\safeaddspace\addspace

\let\rm\rmfamily
\RequirePackage[load-musixtex]{realbookchords}

\let\addspace\safeaddspace

\setcnltx{
  package  = {realbookchords} ,
%  info     = {Typeset Acronyms and other Abbreviations} ,
  authors  = Clemens Niederberger ,
  email    = contact@mychemistry.eu ,
  url      = https://bitbucket.org/cgnieder/realbookchords/ ,
  add-cmds = {
    NewRealBookChords, rbc, RBCsetup, dt, bb, bk, kb, kk, SongTitle, addspace
  } ,
  add-silent-cmds = {
    RealBookTitle, RealBookChords, textordmasculine, mbreve, musixtexaddspace,
    generalmeter, meterfrac, setsign, nobarnumbers, startpiece, afterruleskip,
    NOtes, qa, en, doublebar, Uptext, qlp, Notes, ca, isluru, tslur, ha,
    NOTEs, endpiece
  } ,
  index-setup = { level = \section , noclearpage }
}

% \usepackage{polyglossia}
% \setmainlanguage{english}
% \usepackage[oldstyle,proportional]{libertine}

\setmonofont[Ligatures=TeX,Scale=MatchLowercase]{DejaVu Sans Mono}

\newfontfamily\RealBookTitle{Real Book Title}
\newfontfamily\RealBookChords{RealBook Chords}

\usepackage{ccicons}
\usepackage{csquotes}

\addbibresource{\jobname.bib}

\usepackage{filecontents}
\begin{filecontents}{\jobname.bib}
@online{web:pietsch:fonts,
  author  = {Jochen Pietsch} ,
  title   = {Jazz Fonts} ,
  url     = {http://notation.jochenpietsch.de/index_e.html} ,
  urldate = {2012-05-10}
}
@book{book:realbook,
  author    = {Various} ,
  title     = {The Real Book Volume I} ,
  subtitle  = {C Edition} ,
  publisher = {Hal Leonard Publishing Corporation} ,
  isbn      = {978-0634060380} ,
  date      = {2000-01-01} ,
  edition   = {6th edition}
}
@book{book:newrealbook,
  author    = {Various} ,
  title     = {The New Real Book Volume I} ,
  subtitle  = {C Edition} ,
  publisher = {Ama Verlag} ,
  isbn      = {978-0961470142} ,
  date      = {2009-02-12} ,
  edition   = {1st edition}
}
\end{filecontents}

\begin{document}
\RBCsetup{breve-version=musixtex}

\section{License and Requirements}

\realbookchords\ is placed under the terms of the LaTeX Project Public License,
version 1.3 or later (\url{http://www.latex-project.org/lppl.txt}).
It has the status \enquote{maintained}.

In order to function properly a rather up to date version of the \pkg{l3kernel}
is needed. \realbookchords\ also needs \pkg*{xparse} (part of the \pkg{l3packages}
bundle) and \pkg{fontspec} for proper font support.

\realbookchords\ does not offer the fonts it uses. You need to install them on
your system by yourself and then use this package with \hologo{XeLaTeX} or
\hologo{LuaLaTeX} to be able to use its features.

\section{About}
On his website Jochen Pietsch offered three fonts \cite{web:pietsch:fonts}
which mimic the styles of the famous The Real Book\footnote{The original
  illegal edition is no longer available, though. I am uncertain if the 6th
  edition still uses the same handwritten font.} \cite{book:realbook} and its
successor The New Real Book \cite{book:newrealbook}:

\begin{itemize}
 \item New Real Book Chords
 \item Real Book Title
 \item RealBook Chords
\end{itemize}

\realbookchords\ does \emph{not} offer these fonts. Instead it offers easy to use macros
to use the first font with \hologo{XeLaTeX} or \hologo{LuaLaTeX} if it is installed
on your system. Especially a macro for the intuitive usage of the sometimes rather
hidden symbols for the creation of chord symbols is offered.

\section{Setup}
\realbookchords\ defines a few options which can be globally set using
\begin{commands}
  \command{RBCsetup}[\marg{options}]
\end{commands}
The options follow a key/value system like in many other \LaTeX\ packages.

\section{The Fonts}
Now first let's take a look at the fonts.  They used to be available at
\url{http://notation.jochenpietsch.de/} and are placed under the Creative
Commons Attribution-Noncommercial-Share Alike~3.0 Unported License
\ccbyncsaeu.  I provide them on \realbookchords' BitBucket site
\url{https://bitbucket.org/cgnieder/realbookchords/}. Jochen Pietsch who
designed the fonts says there that he doesn't develop the fonts any further so
we have to live with any shortcomings:
\begin{cnltxquote}[Jochen Pietsch]
  Please note, these fonts are still in beta-stage, so you might note some
  missing characters, or you may need to tweak the layout. Because my interest
  in this project has decreased, I have no plans to do any further
  improvements.
\end{cnltxquote}
The license of the fonts forbids to use them for commercial purposes.  This is
not strictly true for this package but as it is rather useless without them,
well\ldots

This documentation defines new font families for two of the three fonts, the third
one is defined by \realbookchords.
\begin{sourcecode}
  % from the preamble of this document
  \newfontfamily\RealBookTitle{Real Book Title}
  \newfontfamily\RealBookChords{RealBook Chords}
\end{sourcecode}

The package provides this font family switch:
\begin{commands}
  \command{NewRealBookChords}
    activate the font family \enquote{New Real Book Chords}
\end{commands}

\subsection{New Real Book Chords}
The most important font for this package is \enquote{New Real Book Chords}. The
font has only a limited number of characters since its purpose is to build chords.
This also means that it has some unexpected symbols. But have a look for yourself:
\begin{example}
  \normalsize\NewRealBookChords ABCDEFGHIJKLMNOPQRSTUVWXYZ \\
  a cdefghijklmnopqrstuvwxyz 0123456789 \\
  .,:!/*="'\&()[]+-\textordmasculine\% \\
  b\`{a}\ss{} \#{}\~{a}? < > ; {\makeatletter @} \_ \\
  \'{a} \aa{} \"{a} \^{a} \`{A} \'{A} \^{A} \~{A} \"{A} \AA{} \AE{} \"{E} \`{E} \'{E}
\end{example}

\subsection{Real Book Title}
The following example is simply to show you the font, \realbookchords\ does not
use it at all.
\begin{example}
  \normalsize\RealBookTitle ABCDEFGHIJKLMNOPQRSTUVWXYZ \\
  abcdefghijklmnopqrstuvwxyz 01234 6 -/\_
\end{example}

\subsection{RealBook Chords}
The following example is simply to show you the font, \realbookchords\ does not
use it at all.
\begin{example}
  \normalsize\RealBookChords\ ABCDEFG ab s 01 34567 9 \#()+-\textasciicircum
\end{example}

\section{Shortcomings}
Since the font is missing some symbols, for instance the uppercase delta for a
maj7 chord or the striked through o (similar to ø) for a half-diminished chord,
for one thing this package does workarounds and else we have to live with it.

There are plans for the future to extend this package for the use with the more
professional font available here: \url{http://www.jazzfont.com/}. But for that
I'll first have to be able to buy the font and experiment with it a bit.
\emph{Finis coronat opus}.

\section{Typesetting Chords}
It's time to get to the important stuff: the chords. To typeset them there is one
basic command:
\begin{commands}
  \command{rbc}[\oarg{options}\marg{chord specs}]
    The \meta{chord specs} will be explained in detail in the rest of the
    documentation.
\end{commands}

\subsection{Basics}
The basic usage is pretty self-explanatory:
\begin{example}
  \rbc{Ab9} \rbc{Cmi} \rbc{E+} \rbc{Gma9} \rbc{F\#mi9} \rbc{Db13}
\end{example}
Note that this is nearly but not exactly the same as using the font directly:
\begin{example}
  \NewRealBookChords Ab9 Cmi E+ Gma9 F\#mi9 Db13
\end{example}
The font provides the characters \rbc{>} and \rbc{<}. \cs{rbc} replaces the strings
\code{mi} and \code{ma} with them. You have also seen that a \code{b} gives \rbc{b}
for a flat root and \verbcode+\#+ gives \rbc{\#} for a sharp root.

\subsection{Extensions}
One very important aspect of \enquote{jazzy} chords is tensions. The basic tensions
are clear -- just insert the intervall number:
\begin{example}
  \rbc{G7} \rbc{A9} \rbc {F11} \rbc{E13} \rbc{B7+}
\end{example}
Often enough one needs alterated extensions. \realbookchords\ defines macros
to access the characters \rbc{\b} and \rbc{\k} easily:
\begin{commands}
  \command{b}
    \rbc{\b} minor/diminished extension
  \command{f}
    alias for \cs{b}
  \command{k}
    \rbc{\k} major/augmented extension
  \command{s}
    alias for \cs{k}
\end{commands}
Note that these macros are only valid inside \cs{rbc} so that the usual
meaning of, \eg, \cs{b} still holds outside. Now let's see them in action:
\begin{example}
  \rbc{Fmi7(\b5)} \rbc{G7(\k9)} \rbc{Eb7(\b9)} \rbc{Db9(\k11)} \rbc{Cmi7(5-)}
\end{example}

There are also some \enquote{extension descriptions}:
\begin{example}
  \rbc{Cmaj9} \rbc{Bbadd9} \rbc{Absus4} \rbc{Galt} \rbc{F7omit3} \rbc{E\#dim}
  \rbc{Daug}
\end{example}
Since they're not all available as single characters, \realbookchords\ fakes the
missing ones. This results in inconsistent looks. That's why \realbookchords
provides an option so that all six are faked:
\begin{options}
  \keybool{use-fake-symbols}\Default{false}
    switch between original characters and faked ones.
\end{options}
\begin{example}
  \RBCsetup{use-fake-symbols}
  \rbc{Cmaj9} \rbc{Bbadd9} \rbc{Absus4} \rbc{Galt} \rbc{F7omit3} \rbc{E\#dim}
  \rbc{Daug}
\end{example}

\subsection{Double Extensions}
Sometimes more than one alterated extension needs to be indicated. The
\enquote{New Real Book Chords} font provides a number of characters for this
purpose. \realbookchords\ provides five macros to access them easily:
\begin{commands}
  \command{bb}[\marg{intervalls}]
    \verbcode=\rbc{\bb{13,9}}= \rbc{\bb{13,9}}
  \command{ff}
    alias for \cs{bb}
  \command{bk}[\marg{intervalls}]
    \verbcode=\rbc{\bk{13,9}}= \rbc{\bk{13,9}}
  \command{fs}
    alias for \cs{bk}
  \command{bs}
    alias for \cs{bk}
  \command{kb}[\marg{intervalls}]
    \verbcode=\rbc{\kb{13,9}}= \rbc{\kb{13,9}}
  \command{sf}
    alias for \cs{kb}
  \command{sb}
    alias for \cs{kb}
  \command{kk}[\marg{intervalls}]
    \verbcode=\rbc{\kk{13,9}}= \rbc{\kk{13,9}}
  \command{ss}
    alias for \cs{kk}
  \command{dt}[\oarg{alterations>}\marg{intervalls}]
    \verbcode=\dt[bb]{13,9}= \dt[bb]{13,9}
\end{commands}
There are four different combinations of alterations. For each there is a
macro, for some of these there are also aliases since “k” (German: “Kreuz”)
may not be the natural choice for everyone.  The fifth macro also provides a
possibility to access to the for combinations but also enables to only
alterate only one of the two extensions. Unlike the first four macros \cs{dt}
is \emph{not only inside} \cs{rbc} defined but can also be used in normal
text.

All possible \meta{alterations} are shown below:
\begin{example}
  \dt{13,9} \dt[bb]{13,9} \dt[bk]{13,9} \dt[kb]{13,9} \dt[kk]{13,9}
  \dt[B]{13,9} \dt[K]{13,9} \dt[b]{13,9} \dt[k]{13,9}
  
  % second possible representation:
  \dt{13,9} \dt[ff]{13,9} \dt[fs]{13,9} \dt[sf]{13,9} \dt[ss]{13,9}
  \dt[F]{13,9} \dt[S]{13,9} \dt[f]{13,9} \dt[s]{13,9}
  
   % third possible representation:
  \dt{13,9} \dt[bb]{13,9} \dt[bs]{13,9} \dt[sb]{13,9} \dt[ss]{13,9}
  \dt[B]{13,9} \dt[S]{13,9} \dt[b]{13,9} \dt[s]{13,9}
\end{example}

The intervall numbers for both extensions are given seperated with a comma. If the
combination is present as a character like for example \rbc[double-extensions-brackets=false]{\dt{13,9}}
then the character is used else it is faked: \rbc[double-extensions-brackets=false]{\dt{9,6}}

\begin{options}
  \keybool{fake-double-extensions}\Default{false}
    when \code{true} \emph{all} double extension numbers are faked.
  \keybool{double-extensions-brackets}\Default{false}
    enclose the double extensions in brackets or not.
\end{options}
There is a shortcut for the \key{double-extensions-brackets} option:
\cs[rbca]{rbc*}. The brackets can also be set explicitly using \code{[} and
\code{]}.
\begin{example}
  \rbc{Eb7\bk{9,5}} \rbc[double-extensions-brackets=false]{C\dt{9,6}}
  \rbc*{C\dt{9,6}} \rbc*{C[\bk{13,9}]}
\end{example}

\subsection{Bass Notes}
If you want to indicate a different bass note for the chord you simply seperate the
main chord from the bass note with a slash. Please note: if the bass note is a
flat or sharp note you have to enclose it in braces:
\begin{example}
  \rbc{Cmi7/{Bb}} \rbc{C7/G} \rbc{C/G} \rbc{E7/{G\#}}
\end{example}

\subsection{Disabling the Parsing}
If you don't want all the parsing of the \cs{rbc} command but simply access the
\enquote{New Real Book Chords} font you can of course use the font family switch
presented earlier:
\begin{example}
  \NewRealBookChords Ama7(omit3) [major]
\end{example}

You can also use the following option:
\begin{options}
  \keybool{parse}\Default{true}
    switch off the parsing.
\end{options}
\begin{example}
  \texttt{parse=true}: \rbc{Ama7(omit3) [major]} \\
  \texttt{parse=false}: \rbc[parse=false]{Ama7(omit3) [major]}
\end{example}

\section{Song Titles}
% TODO Optionen beschreiben
\begin{commands}
  \command{SongTitle}[\oarg{left}\marg{center}\oarg{right}]
    A centered title with options to put text to the left and the right of it.
\end{commands}
\begin{options}
  \keyval{songtitle-format-left}{code}%
  \Default{\cs*{NewRealBookChords}\cs*{footnotesize}}
    Format of the text to the left.
  \keyval{songtitle-format-center}{code}%
  \Default{\cs*{NewRealBookChords}\cs*{Large}\cs*{centering}}
    Format of the text in the center.
  \keyval{songtitle-format-right}{code}%
  \Default{\cs*{NewRealBookChords}\cs*{footnotesize}}
    Format of the text to the right.
  \keychoice{songtitle-pos-left}{l,c,r}\Default{l}
    Alignment of the text to the left.
  \keychoice{songtitle-pos-right}{l,c,r}\Default{l}
    Alignment of the text to the right.
\end{options}

As an example: the following code \ldots
\begin{sourcecode}
  \SongTitle{Begin The Beguine}
  \par\vskip\baselineskip
  \SongTitle[Medium Swing Tempo]{Begin The Beguine}
  \par\vskip\baselineskip
  \SongTitle{Begin The Beguine}[Cole Porter\\arr.: Jerry Sears]
  \par\vskip\baselineskip 
  \RBCsetup{songtitle-pos-right=r}
  \SongTitle{Begin The Beguine}[Cole Porter\\arr.: Jerry Sears]
\end{sourcecode}
\ldots{} produces this:

\begingroup
  \SongTitle{Begin The Beguine}
  \par\vskip\baselineskip
  \SongTitle[Medium Swing Tempo]{Begin The Beguine}
  \par\vskip\baselineskip
  \SongTitle{Begin The Beguine}[Cole Porter\\arr.: Jerry Sears]
  \par\vskip\baselineskip 
  \RBCsetup{songtitle-pos-right=r}
  \SongTitle{Begin The Beguine}[Cole Porter\\arr.: Jerry Sears]
\endgroup

\section{See it in Action}
Let's use the \cs{rbc} command together with \pkg{musixtex} for a real example.
Both \pkg{fontspec} which is loaded be \realbookchords\ and \pkg{musixtex}
define the command \cs{breve}. The definition by \pkg{fontspec} takes place
\verbcode=\AtBeginDocument=.

A similar problem arises when \pkg{musixtex} is used together with \pkg{biblatex}:
they both define \cs{addspace}.

You can do something like the following to get them to work together:
\begin{sourcecode}
  \documentclass{article}
  \usepackage{musixtex}
  \let\mbreve\breve
  \let\breve\relax
  \usepackage{realbookchords}
  \begin{document}
  do stuff, e.g. restore the `musixtex' definition of \breve
  \end{document}
\end{sourcecode}

Or -- even easier -- load \realbookchords\ with the \key{load-musixtex} option and
let it handle things. However, this doesn't always work as expected, yet.
\begin{sourcecode}
  \documentclass{article}
  \usepackage[load-musixtex]{realbookchords}
  \begin{document}
  do stuff, e.g. restore the `musixtex' definition of \breve
  \end{document}
\end{sourcecode}
Note that this option should be used as a package option and \emph{not} via
\cs{RBCsetup}.

If you use this option \realbookchords\ loads \pkg{musixtex} and sets \cs{breve}
to the meaning of \pkg{fontspec}. You have the possibility to access \pkg{musixtex}'s
version:
\begin{commands}
  \command{musixtexbreve}
    \pkg{musixtex}'s original definition.
  \command{fontspecbreve}
    \pkg{fontspec}'s original definition.
  \command{musixtexaddspace}
    \realbookchords\ undefines \pkg{musixtex}'s \cs{addspace} but stores the
    definition in \cs{musixtexaddspace}.  You'll probably need to restore
    \pkg{musixtex}'s definition.  If you don't use \pkg{biblatex} in the
    same document there shouldn't be a problem.
\end{commands}
\begin{options}
  \keychoice{breve-version}{fontspec,musixtex}
    sets \cs{breve} to the definition of the specified package.
  \keychoice{addspace-version}{musixtex}
    restores \pkg{musixtex}'s definition.
\end{options}

Let's take the first eight bars of Kenny Durham's \enquote{Blue Bossa} as an example:

\begin{sourcecode}
  \begin{music}
    \let\addspace\musixtexaddspace
    \parindent0pt \generalmeter{\meterfrac44}\setsign{1}{-3}\nobarnumbers
    \SongTitle{Blue Bossa}[\\Kenny Durham]
    \startpiece
    \addspace{.5\afterruleskip}%
    \NOtes\qa g\en
    \doublebar % 1
    \NOtes\Uptext{\rbc{Cmi6}}\qlp n\en
    \Notes\ca l\en
    \Notes\qa k\en
    \Notes\isluru0j\ca j\en
    \bar % 2
    \NOtes\tslur0j\hlp j\en
    \NOtes\qa i\en
    \bar % 3
    \NOTes\Uptext{\rbc{Fmi7}}\ha h\en
    \Notes\qlp n\en
    \Notes\isluru0m\ca m\en
    \bar % 4
    \NOTEs\tslur0m\wh m\en
    \bar % 5
    \NOtes\Uptext{\rbc{Dmi7(\b5)}}\qlp m\en
    \Notes\ca{lk}\qa j\en
    \Notes\isluru0i\ca i\en
    \bar % 6
    \NOTEs\Uptext{\rbc{G7(\k5)}}\tslur0i\hlp i\en
    \Notes\qa h\en
    \bar % 7
    \NOTes\Uptext{\rbc{Cmi6}}\ha g\en
    \Notes\qlp m\isluru0l\ca l\en
    \bar % 8
    \NOTEs\tslur0l\wh l\en
    \endpiece
  \end{music}
\end{sourcecode}

\begin{music}
 \let\addspace\musixtexaddspace
 \parindent0pt \generalmeter{\meterfrac44}\setsign{1}{-3}\nobarnumbers
 \SongTitle{Blue Bossa}[\\Kenny Durham]
 \startpiece
 \addspace{.5\afterruleskip}%
 \NOtes\qa g\en
 \doublebar % 1
 \NOtes\Uptext{\rbc{Cmi6}}\qlp n\en
 \Notes\ca l\en
 \Notes\qa k\en
 \Notes\isluru0j\ca j\en
 \bar % 2
 \NOtes\tslur0j\hlp j\en
 \NOtes\qa i\en
 \bar % 3
 \NOTes\Uptext{\rbc{Fmi7}}\ha h\en
 \Notes\qlp n\en
 \Notes\isluru0m\ca m\en
 \bar % 4
 \NOTEs\tslur0m\wh m\en
 \bar % 5
 \NOtes\Uptext{\rbc{Dmi7(\b5)}}\qlp m\en
 \Notes\ca{lk}\qa j\en
 \Notes\isluru0i\ca i\en
 \bar % 6
 \NOTEs\Uptext{\rbc{G7(\k5)}}\tslur0i\hlp i\en
 \Notes\qa h\en
 \bar % 7
 \NOTes\Uptext{\rbc{Cmi6}}\ha g\en
 \Notes\qlp m\isluru0l\ca l\en
 \bar % 8
 \NOTEs\tslur0l\wh l\en
 \endpiece
\end{music}

\end{document}
