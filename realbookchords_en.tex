% arara: musixtex: { engine: xelatex }
% arara: biber
% arara: xelatex
% arara: xelatex
% --------------------------------------------------------------------------
% the REALBOOKCHORDS package
% 
%   Typesetting jazz chords the `Real Book' way
% 
% 2013/04/25
% --------------------------------------------------------------------------
% Clemens Niederberger
% Web:    https://bitbucket.org/cgnieder/realbookchords/
% E-Mail: contact@mychemistry.eu
% --------------------------------------------------------------------------
% Copyright 2012--2013 Clemens Niederberger
% 
% This work may be distributed and/or modified under the
% conditions of the LaTeX Project Public License, either version 1.3
% of this license or (at your option) any later version.
% The latest version of this license is in
%   http://www.latex-project.org/lppl.txt
% and version 1.3 or later is part of all distributions of LaTeX
% version 2005/12/01 or later.
% 
% This work has the LPPL maintenance status `maintained'.
% 
% The Current Maintainer of this work is Clemens Niederberger.
% --------------------------------------------------------------------------
% The realbookchords package consists of the files
%  - realbookchords.sty,
%  - realbookchords_en.tex,
%  - realbookchords_en.pdf,
%  - README
% --------------------------------------------------------------------------
% If you have any ideas, questions, suggestions or bugs to report, please
% feel free to contact me.
% --------------------------------------------------------------------------
\let\rm\rmfamily
\RequirePackage[load-musixtex]{realbookchords}
\documentclass[toc=index,toc=bib]{cnpkgdoc}
\docsetup{
  pkg      = realbookchords ,
  code-box = {
    backgroundcolor  = gray!7!white ,
    skipbelow        = .6\baselineskip plus .5ex minus .5ex ,
    skipabove        = .6\baselineskip plus .5ex minus .5ex ,
    roundcorner      = 3pt ,
  }
}
\usepackage{polyglossia}
\setmainlanguage{english}
\usepackage[oldstyle,proportional]{libertine}
\usepackage{libertinehologopatch}

\setmonofont[Ligatures=TeX,Scale=MatchLowercase]{Bitstream Vera Sans Mono}

\usepackage{fnpct}
\cnpkgcolors{
  main   => cnpkgred ,
  key    => yellow!40!brown ,
  module => cnpkgblue ,
  link   => black!90
}

\renewcommand*\othersectionlevelsformat[3]{%
  \textcolor{main}{#3\autodot}\enskip}
\renewcommand*\partformat{%
  \textcolor{main}{\partname~\thepart\autodot}}

\addcmds{
  addspace, afterruleskip, AtBeginDocument,
  b, bar, bb, bk, breve,
  ca, doublebar, dt, en, endpiece, fontspecbreve,
  generalmeter, ha, hlp, isluru, k, kb, kk,
  mbreve, meterfrac, musixtexaddspace, musixtexbreve,
  newfontfamily, NewRealBookChords, nobarnumbers,
  notes,Notes,NOtes,NOTes,NOTEs,
  qlp,qu, rbc, RBCsetup, RealBookChords, RealBookTitle,
  setsign, SongTitle, startpiece,
  textordmasculine, tslur, Uptext
}

\newfontfamily\RealBookTitle{Real Book Title}
\newfontfamily\RealBookChords{RealBook Chords}

\usepackage{ccicons}
\usepackage{csquotes}

\usepackage[backend=biber,style=alphabetic]{biblatex}
\addbibresource{\jobname.bib}

\usepackage{filecontents}
\begin{filecontents}{\jobname.bib}
@online{web:pietsch:fonts,
  author  = {Jochen Pietsch} ,
  title   = {Jazz Fonts} ,
  url     = {http://notation.jochenpietsch.de/index_e.html} ,
  urldate = {2012-05-10}
}
@book{book:realbook,
  author    = {Various} ,
  title     = {The Real Book Volume I} ,
  subtitle  = {C Edition} ,
  publisher = {Hal Leonard Publishing Corporation} ,
  isbn      = {978-0634060380} ,
  date      = {2000-01-01} ,
  edition   = {6th edition}
}
@book{book:newrealbook,
  author    = {Various} ,
  title     = {The New Real Book Volume I} ,
  subtitle  = {C Edition} ,
  publisher = {Ama Verlag} ,
  isbn      = {978-0961470142} ,
  date      = {2009-02-12} ,
  edition   = {1st edition}
}
\end{filecontents}

\usepackage{imakeidx}
\begin{filecontents*}{\jobname.ist}
 heading_prefix "{\\bfseries "
 heading_suffix "\\hfil}\\nopagebreak\n"
 headings_flag  1
 delim_0 "\\dotfill\\hyperpage{"
 delim_1 "\\dotfill\\hyperpage{"
 delim_2 "\\dotfill\\hyperpage{"
 delim_r "}\\textendash\\hyperpage{"
 delim_t "}"
 suffix_2p "\\nohyperpage{\\,f.}"
 suffix_3p "\\nohyperpage{\\,ff.}"
\end{filecontents*}
\indexsetup{othercode=\footnotesize}
\makeindex[options={-s \jobname.ist},intoc,columns=3,columnsep=1em]

\let\verb\lstinline

\newcommand*\Default[1]{%
  \hfill\llap
    {%
      \ifblank{#1}%
        {(initially~empty)}%
        {Default:~\code{#1}}%
    }%
  \newline
}
\usepackage{marginnote,ragged2e}
\makeatletter
\providecommand*\sinceversion[1]{%
  \@bsphack
  \marginnote{%
    \footnotesize\sffamily\RaggedRight
    \textcolor{black!75}{Introduced in version~#1}}%
  \@esphack}
\providecommand*\changedversion[1]{%
  \@bsphack
  \marginnote{%
    \footnotesize\sffamily\RaggedRight
    \textcolor{black!75}{Changed in version~#1}}%
  \@esphack}
\makeatother

\begin{document}
\RBCsetup{breve-version=musixtex}

\section{License and Requirements}

\realbookchords is placed under the terms of the LaTeX Project Public License,
version 1.3 or later (\url{http://www.latex-project.org/lppl.txt}).
It has the status \enquote{maintained}.

In order to function properly a rather up to date version of the \paket{l3kernel}
is needed. \realbookchords also needs \paket*{xparse} (part of the \paket{l3packages}
bundle) and \paket{fontspec} for proper font support.

\realbookchords does not offer the fonts it uses. You need to install them on
your system by yourself and then use this package with \hologo{XeLaTeX} or
\hologo{LuaLaTeX} to be able to use its features.

\section{About}
On his website Jochen Pietsch offers three fonts \cite{web:pietsch:fonts} which
mimic the styles of the famous The Real Book\footnote{The original illegal edition
is no longer available, though. I am uncertain if the 6th edition still uses the
same handwritten font.} \cite{book:realbook} and its successor The New Real Book
\cite{book:newrealbook}:

\begin{itemize}
 \item New Real Book Chords
 \item Real Book Title
 \item RealBook Chords
\end{itemize}

\realbookchords does \emph{not} offer these fonts. Instead it offers easy to use macros
to use the first font with \hologo{XeLaTeX} or \hologo{LuaLaTeX} if it is installed
on your system. Especially a macro for the intuitive usage of the sometimes rather
hidden symbols for the creation of chord symbols is offered.

\section{Setup}
\realbookchords defines a few options which can be globally set using
\begin{beschreibung}
 \Befehl{RBCsetup}{<options>}
\end{beschreibung}
The options follow a key/value system like in many other \LaTeX\ packages.

\section{The Fonts}
Now first let's take a look at the fonts. You can get them\footnote{Just in case
the website isn't reachable any more or the fonts are no longer distributed
there you can also get a copy from me. In this case just send me an email.}
at \url{http://notation.jochenpietsch.de/}. They are placed under the Creative
Commons Attribution-Noncommercial-Share Alike~3.0 Unported License \ccbyncsaeu.
Jochen Pietsch who designed the fonts says there that he doesn't develop the fonts
any further so we have to live with any shortcomings:

\begin{zitat}[Jochen Pietsch]
 Please note, these fonts are still in beta-stage, so you might note some missing
 characters, or you may need to tweak the layout. Because my interest in this
 project has decreased, I have no plans to do any further improvements.
\end{zitat}

The license of the fonts forbids to use them for commercial purposes. This is not
strictly true for this package but as it is rather useless without them, well\ldots

This documentation defines new font families for two of the three fonts, the third
one is defined by \realbookchords.
\begin{beispiel}
 % from the preamble of this document
 \newfontfamily\RealBookTitle{Real Book Title}
 \newfontfamily\RealBookChords{RealBook Chords}
\end{beispiel}

The package provides this font family switch:
\begin{beschreibung}
 \Befehl{NewRealBookChords}\newline
   activate the font family \enquote{New Real Book Chords}
\end{beschreibung}

\subsection{New Real Book Chords}
The most important font for this package is \enquote{New Real Book Chords}. The
font has only a limited number of characters since its purpose is to build chords.
This also means that it has some unexpected symbols. But have a look for yourself:
\begin{beispiel}
 \normalsize\NewRealBookChords ABCDEFGHIJKLMNOPQRSTUVWXYZ \\
 a cdefghijklmnopqrstuvwxyz 0123456789 \\
 .,:!/*="'\&()[]+-\textordmasculine\% \\
 b\`{a}\ss{} \#{}\~{a}? < > ; {\makeatletter @} \_ \\
 \'{a} \aa{} \"{a} \^{a} \`{A} \'{A} \^{A} \~{A} \"{A} \AA{} \AE{} \"{E} \`{E} \'{E}
\end{beispiel}

\subsection{Real Book Title}
The following example is simply to show you the font, \realbookchords does not
use it at all.
\begin{beispiel}
 \normalsize\RealBookTitle ABCDEFGHIJKLMNOPQRSTUVWXYZ \\
 abcdefghijklmnopqrstuvwxyz 01234 6 -/\_
\end{beispiel}

\subsection{RealBook Chords}
The following example is simply to show you the font, \realbookchords does not
use it at all.
\begin{beispiel}
 \normalsize\RealBookChords ABCDEFG ab s 01 34567 9 \#()+-\textasciicircum
\end{beispiel}

\section{Shortcomings}
Since the font is missing some symbols, for instance the uppercase delta for a
maj7 chord or the striked through o (similar to ø) for a half-diminished chord,
for one thing this package does workarounds and else we have to live with it.

There are plans for the future to extend this package for the use with the more
professional font available here: \url{http://www.jazzfont.com/}. But for that
I'll first have to be able to buy the font and experiment with it a bit.
\emph{Finis coronat opus}.

\section{Typesetting Chords}
It's time to get to the important stuff: the chords. To typeset them there is one
basic command:
\begin{beschreibung}
 \Befehl{rbc}[<options>]{<chord specs>}
\end{beschreibung}
The \ma{<chord specs>} will be explained in detail in the rest of the documentation.

\subsection{Basics}
The basic usage is pretty self-explanatory:
\begin{beispiel}
 \rbc{Ab9} \rbc{Cmi} \rbc{E+} \rbc{Gma9} \rbc{F\#mi9} \rbc{Db13}
\end{beispiel}
Note that this is nearly but not exactly the same as using the font directly:
\begin{beispiel}
 \NewRealBookChords Ab9 Cmi E+ Gma9 F\#mi9 Db13
\end{beispiel}
The font provides the characters \rbc{>} and \rbc{<}. \cmd{jc} replaces the strings
\code{mi} and \code{ma} with them. You have also seen that a \code{b} gives \rbc{b}
for a flat root and \verb+\#+ gives \rbc{\#} for a sharp root.

\subsection{Extensions}
One very important aspect of \enquote{jazzy} chords is tensions. The basic tensions
are clear -- just insert the intervall number:
\begin{beispiel}
 \rbc{G7} \rbc{A9} \rbc {F11} \rbc{E13} \rbc{B7+}
\end{beispiel}
Often enough one needs alterated extensions. \realbookchords defines two macros
to access the characters \rbc{\b} and \rbc{\k} easily:
\begin{beschreibung}
 \Befehl{b}\newline
   minor/diminished extension
 \Befehl{k}\newline
   major/augmented extension
\end{beschreibung}
 Let's see them in action:
\begin{beispiel}
  \rbc{Fmi7(\b5)} \rbc{G7(\k9)} \rbc{Eb7(\b9)} \rbc{Db9(\k11)} \rbc{Cmi7(5-)}
\end{beispiel}

There are also six \enquote{extension descriptions}:
\begin{beispiel}
 \rbc{Cmaj9} \rbc{Bbadd9} \rbc{Gsus4} \rbc{Ealt} \rbc{F7omit3} \rbc{D\#dim} \\
\end{beispiel}
Since they're not all available as single characters, \realbookchords fakes the
missing ones. This results in inconsistent looks. That's why \realbookchords
provides an option so that all six are faked:
\begin{beschreibung}
 \Option{use-fake-symbols}{\default{true}|false}\Default{false}
   switch between original characters and faked ones.
\end{beschreibung}
\begin{beispiel}
 \RBCsetup{use-fake-symbols}
 \rbc{Cmaj9} \rbc{Bbadd9} \rbc{Gsus4} \rbc{Ealt} \rbc{F7omit3} \rbc{D\#dim}
\end{beispiel}

\subsection{Double Extensions}
Sometimes more than one alterated extension needs to be indicated. The
\enquote{New Real Book Chords} font provides a number of characters for this
purpose. \realbookchords provides five macros to access them easily:
\begin{beschreibung}
 \Befehl{bb}{<intervalls>}\newline
   \verb=\rbc{\bb{13,9}}= \rbc{\bb{13,9}}
 \Befehl{bk}{<intervalls>}\newline
   \verb=\rbc{\bk{13,9}}= \rbc{\bk{13,9}}
 \Befehl{kb}{<intervalls>}\newline
   \verb=\rbc{\kb{13,9}}= \rbc{\kb{13,9}}
 \Befehl{kk}{<intervalls>}\newline
   \verb=\rbc{\kk{13,9}}= \rbc{\kk{13,9}}
 \Befehl{dt}[<alterations>]{<intervalls>}\newline
   \verb=\dt[bb]{13,9}= \dt[bb]{13,9}
\end{beschreibung}
There are four different combinations of alterations. For each there is a macro.
The fifth macro also provides a possibility to access to the for combinations but
also enables to only alterate only one of the two extensions. Unlike the first
four macros \cmd{dt} is \emph{not only inside} \cmd{rbc} defined but can also be
used in normal text.

All possible \oa{<alterations>} are shown below:
\begin{beispiel}
 \dt{13,9} \dt[bb]{13,9} \dt[bk]{13,9} \dt[kb]{13,9} \dt[kk]{13,9} \dt[B]{13,9} \dt[K]{13,9} \dt[b]{13,9} \dt[k]{13,9}
\end{beispiel}

The intervall numbers for both extensions are given seperated with a comma. If the
combination is present as a character like for example \rbc[double-extensions-brackets=false]{\dt{13,9}}
then the character is used else it is faked: \rbc[double-extensions-brackets=false]{\dt{9,6}}

\begin{beschreibung}
 \Option{fake-double-extensions}{\default{true}|false}\Default{false}
   when \code{true} \emph{all} double extension numbers are faked.
 \Option{double-extensions-brackets}{\default{true}|false}\Default{false}
   enclose the double extensions in brackets or not.
\end{beschreibung}
There is a shortcut for the \key{double-extensions-brackets} option: \cmd[rbca]{rbc*}. The brackets can also be set explicitly using \code{[} and \code{]}.
\begin{beispiel}
 \rbc{Eb7\bk{9,5}} \rbc[double-extensions-brackets=false]{C\dt{9,6}} \rbc*{C\dt{9,6}} \rbc*{C[\bk{13,9}]}
\end{beispiel}

\subsection{Bass Notes}
If you want to indicate a different bass note for the chord you simply seperate the
main chord from the bass note with a slash. Please note: if the bass note is a
flat or sharp note you have to enclose it in braces:
\begin{beispiel}
 \rbc{Cmi7/{Bb}} \rbc{C7/G} \rbc{C/G} \rbc{E7/{G\#}}
\end{beispiel}

\subsection{Disabling the Parsing}
If you don't want all the parsing of the \cmd{rbc} command but simply access the
\enquote{New Real Book Chords} font you can of course use the font family switch
presented earlier:
\begin{beispiel}
 \NewRealBookChords Ama7(omit3) [major]
\end{beispiel}

You can also use the following option:
\begin{beschreibung}
 \Option{parse}{\default{true}|false}\Default{true}
   switch off the parsing.
\end{beschreibung}
\begin{beispiel}
 \texttt{parse=true}: \rbc{Ama7(omit3) [major]} \\
 \texttt{parse=false}: \rbc[parse=false]{Ama7(omit3) [major]}
\end{beispiel}

\section{Song Titles}
% TODO Optionen beschreiben
\begin{beschreibung}
 \Befehl{SongTitle}[<left>]{<center>}\oa{<right}\newline
   A centered title with options to put text to the left and the right of it.
 \Option{songtitle-format-left}{<code>}\Default{\cmd*{NewRealBookChords}\cmd*{footnotesize}}
   Format of the text to the left.
 \Option{songtitle-format-center}{<code>}\Default{\cmd*{NewRealBookChords}\cmd*{Large}\cmd*{centering}}
   Format of the text in the center.
 \Option{songtitle-format-right}{<code>}\Default{\cmd*{NewRealBookChords}\cmd*{footnotesize}}
   Format of the text to the right.
 \Option{songtitle-pos-left}{l|c|r}\Default{l}
   Alignment of the text to the left.
 \Option{songtitle-pos-right}{l|c|r}\Default{l}
   Alignment of the text to the right.
\end{beschreibung}

As an example: the following code \ldots
\begin{beispiel}[code only]
 \SongTitle{Begin The Beguine}
 \par\vskip\baselineskip
 \SongTitle[Medium Swing Tempo]{Begin The Beguine}
 \par\vskip\baselineskip
 \SongTitle{Begin The Beguine}[Cole Porter\\arr.: Jerry Sears]
 \par\vskip\baselineskip 
 \RBCsetup{songtitle-pos-right=r}
 \SongTitle{Begin The Beguine}[Cole Porter\\arr.: Jerry Sears]
\end{beispiel}
\ldots{} produces this:

\begingroup
 \SongTitle{Begin The Beguine}
 \par\vskip\baselineskip
 \SongTitle[Medium Swing Tempo]{Begin The Beguine}
 \par\vskip\baselineskip
 \SongTitle{Begin The Beguine}[Cole Porter\\arr.: Jerry Sears]
 \par\vskip\baselineskip 
 \RBCsetup{songtitle-pos-right=r}
 \SongTitle{Begin The Beguine}[Cole Porter\\arr.: Jerry Sears]
\endgroup

\section{See it in Action}
Let's use the \cmd{rbc} command together with \paket{musixtex} for a real example.
Both \paket{fontspec} which is loaded be \realbookchords and \paket{musixtex}
define the command \cmd{breve}. The definition by \paket{fontspec} takes place
\verb=\AtBeginDocument=.

A similar problem arises when \paket{musixtex} is used together with \paket{biblatex}:
they both define \cmd{addspace}.

You can do something like the following to get them to work together:
\begin{beispiel}[code only]
 \documentclass{article}
 \usepackage{musixtex}
 \let\mbreve\breve
 \let\breve\relax
 \usepackage{realbookchords}
 \begin{document}
  do stuff, e.g. restore the `musixtex' definition of \breve
 \end{document}
\end{beispiel}

Or -- even easier -- load \realbookchords with the \key{load-musixtex} option and
let it handle things. However, this doesn't always work as expected, yet.
\begin{beispiel}[code only]
 \documentclass{article}
 \usepackage[load-musixtex]{realbookchords}
 \begin{document}
  do stuff, e.g. restore the `musixtex' definition of \breve
 \end{document}
\end{beispiel}
Note that this option should be used as a package option and \emph{not} via
\cmd{RBCsetup}.

If you use this option \realbookchords loads \paket{musixtex} and sets \cmd{breve}
to the meaning of \paket{fontspec}. You have the possibility to access \paket{musixtex}'s
version:
\begin{beschreibung}
 \Befehl{musixtexbreve}\newline
   \paket{musixtex}'s original definition.
 \Befehl{fontspecbreve}\newline
   \paket{fontspec}'s original definition.
 \Option{breve-version}{fontspec|musixtex}\newline
   sets \cmd{breve} to the definition of the specified package.
 \Befehl{musixtexaddspace}\newline
   \realbookchords undefines \paket{musixtex}'s \cmd{addspace} but stores the
   definition in \cmd{musixtexaddspace}. You'll probably need to restore
   \paket{musixtex}'s definition. If you don't use \paket{biblatex} in the
   same document there shouldn't be a problem.
 \Option{addspace-version}{musixtex}\newline
   restores \paket{musixtex}'s definition.
\end{beschreibung}

Let's take the first eight bars of Kenny Durham's \enquote{Blue Bossa} as an example:

\begin{beispiel}[code only]
\begin{music}
 \let\addspace\musixtexaddspace
 \parindent0pt \generalmeter{\meterfrac44}\setsign{1}{-3}\nobarnumbers
 \SongTitle{Blue Bossa}[\\Kenny Durham]
 \startpiece
 \addspace{.5\afterruleskip}%
 \NOtes\qa g\en
 \doublebar % 1
 \NOtes\Uptext{\rbc{Cmi6}}\qlp n\en
 \Notes\ca l\en
 \Notes\qa k\en
 \Notes\isluru0j\ca j\en
 \bar % 2
 \NOtes\tslur0j\hlp j\en
 \NOtes\qa i\en
 \bar % 3
 \NOTes\Uptext{\rbc{Fmi7}}\ha h\en
 \Notes\qlp n\en
 \Notes\isluru0m\ca m\en
 \bar % 4
 \NOTEs\tslur0m\wh m\en
 \bar % 5
 \NOtes\Uptext{\rbc{Dmi7(\b5)}}\qlp m\en
 \Notes\ca{lk}\qa j\en
 \Notes\isluru0i\ca i\en
 \bar % 6
 \NOTEs\Uptext{\rbc{G7(\k5)}}\tslur0i\hlp i\en
 \Notes\qa h\en
 \bar % 7
 \NOTes\Uptext{\rbc{Cmi6}}\ha g\en
 \Notes\qlp m\isluru0l\ca l\en
 \bar % 8
 \NOTEs\tslur0l\wh l\en
 \endpiece
\end{music}
\end{beispiel}

\begin{music}
 \let\addspace\musixtexaddspace
 \parindent0pt \generalmeter{\meterfrac44}\setsign{1}{-3}\nobarnumbers
 \SongTitle{Blue Bossa}[\\Kenny Durham]
 \startpiece
 \addspace{.5\afterruleskip}%
 \NOtes\qa g\en
 \doublebar % 1
 \NOtes\Uptext{\rbc{Cmi6}}\qlp n\en
 \Notes\ca l\en
 \Notes\qa k\en
 \Notes\isluru0j\ca j\en
 \bar % 2
 \NOtes\tslur0j\hlp j\en
 \NOtes\qa i\en
 \bar % 3
 \NOTes\Uptext{\rbc{Fmi7}}\ha h\en
 \Notes\qlp n\en
 \Notes\isluru0m\ca m\en
 \bar % 4
 \NOTEs\tslur0m\wh m\en
 \bar % 5
 \NOtes\Uptext{\rbc{Dmi7(\b5)}}\qlp m\en
 \Notes\ca{lk}\qa j\en
 \Notes\isluru0i\ca i\en
 \bar % 6
 \NOTEs\Uptext{\rbc{G7(\k5)}}\tslur0i\hlp i\en
 \Notes\qa h\en
 \bar % 7
 \NOTes\Uptext{\rbc{Cmi6}}\ha g\en
 \Notes\qlp m\isluru0l\ca l\en
 \bar % 8
 \NOTEs\tslur0l\wh l\en
 \endpiece
\end{music}

\appendix

\printbibliography

\addsec{Implementation}
\implementation[linerange={35-1000},firstnumber=35]

\indexprologue{\noindent Section titles are indicated \textbf{bold}, packages
\textsf{sans serif}, commands \code{\textbackslash\textcolor{code}{brown}}
 and options \textcolor{key}{\code{yellow}}.\par\bigskip}
\printindex

\end{document}